\documentclass[a4paper,11pt]{article}
  \usepackage{cmap}
  \usepackage[english,russian]{babel}
  \usepackage[utf8]{inputenc}
  \usepackage{indentfirst}
  \topmargin=-2cm
  \frenchspacing

  \author{И.\,А.\,Новиков, кафедра АФТИ ФФ НГУ, гр.\,7305\\ Руководитель: Д.\,А.\,Гладкий, Софтлаб-Нск}
  \title{<<Система моделирования деформаций неупругих тел в реальном времени>>\\ Аннотация}
\begin{document}
  \maketitle
  \thispagestyle{empty}
  \paragraph{Введение.}
    В подсистемах моделирования физики современных автомобильных симуляторов большое внимание
    уделяется деформации моделей в результате столкновений с окружением. Визуально
    правдоподобные деформации повышают реализм игрового процесса, при этом детальная
    физическая достоверность не требуется. Достаточно распространён подход, при
    котором используются предварительно рассчитанные деформированные состояния модели.
    Такой подход имеет некоторые недостатки.
    Во-первых, это ограничивает разнообразие возможных деформаций, что негативно сказывается на
    реализме.  Во-вторых, такая система требует значительного времени на предварительные расчёты после изменения
    модели, что создаёт существенные неудобства для дизайнеров.
  \paragraph{Цели.}
    Целью работы является разработка системы, которая будет моделировать деформации в реальном
    времени, не требуя значительных предварительных расчётов, и будет легко интегрироваться в существующий
    физический движок.
  \paragraph{Задачи.}
    Для достижения цели необходимо решить следующие задачи:
    \begin{enumerate}
      \item Реализовать алгоритм моделирования деформаций неупругих тел, заданных полигональной сеткой.
      \item Предоставить интерфейс для взаимодействия с остальной частью подсистемы моделирования
        физики (получение от неё информации о столкновениях и приложенных силах и сообщение ей о
        сильных локальных деформациях) и графической подсистемой (обновление отображаемой сетки).
      \item Предусмотреть в алгоритме возможность использования параллельных вычислений.
    \end{enumerate}
\end{document}

