\documentclass[a4paper,11pt,twocolumn]{article}
  \usepackage{cmap}
  \usepackage[english,russian]{babel}
  \usepackage[utf8]{inputenc}
  \usepackage{indentfirst}
  \frenchspacing

  \author{И.\,А.\,Новиков, кафедра АФТИ ФФ НГУ, гр.\,7305\\ Руководитель: Д.\,А.\,Гладкий, Софтлаб-Нск}
  \title{<<Система моделирования деформаций пластичных тел в реальном времени>>\\ Аннотация}
\begin{document}
  \maketitle
  \thispagestyle{empty}
  \paragraph{Введение.}
    В физических подсистемах современных компьютерных игр в жанре автосимулятора большое внимание
    уделяется моделированию деформаций моделей в результате столкновений с окружением. Визуально
    правдоподобные деформации повышают реализм игрового процесса, в то время как детальная
    физическая достоверность не требуется. Разработанная в SoftLab-Nsk система моделирования
    деформаций, применяемая в игре <<Дальнобойщики~3: Покорение Америки>>, имеет ряд недостатков.
    Она использует предварительно рассчитанные деформированные состояния модели. Во-первых, это
    ограничивает разнообразие возможных деформаций, что негативно сказывается на реализме.
    Во-вторых, такая система требует значительного времени на предрасчёты после изменения модели,
    что создаёт существенные неудобства для дизайнеров.
  \paragraph{Цели.}
    В работе ставится цель разработать систему, которая будет моделировать деформации в реальном
    времени, не будет требовать значительных предрасчётов и будет легко интегрироваться в
    существующий физический движок.
  \newpage
  \paragraph{Задачи.}
    Для достижения цели будет решены следующие задачи:
    \begin{enumerate}
      \item Реализовать алгоритм моделирования деформаций пластичных тел, заданных полигональной сеткой.
      \item Предоставить интерфейс для взаимодействия с остальной частью физической подсистемы
        (получение от неё информации о столкновениях и приложенных силах и сообщение ей о
        сильных локальных деформациях) и графической подсистемой (обновление отображаемой сетки).
      \item Предусмотреть применение системы на компьютерах с многоядерными процессорами: процесс
        вычислений необходимо разбивать на задачи, которые могут быть выполнены параллельно.
    \end{enumerate}
\end{document}

