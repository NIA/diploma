\documentclass[a4paper,11pt]{article}
  \usepackage{cmap}
  \usepackage[english,russian]{babel}
  \usepackage[utf8]{inputenc}
  \usepackage{indentfirst}
  \addtolength{\topmargin}{-2cm}
  \addtolength{\textheight}{4cm}
  \addtolength{\oddsidemargin}{-0.5cm}
  \addtolength{\evensidemargin}{-0.5cm}
  \addtolength{\textwidth}{1cm}
  \frenchspacing

  \author{И.\,А.\,Новиков, кафедра АФТИ ФФ НГУ, гр.\,7305\\ Руководитель: Д.\,А.\,Гладкий, Софтлаб-Нск}
  \title{<<Система моделирования деформаций неупругих тел в реальном времени>>\\\itshape Аннотация}
\begin{document}
  \maketitle
  \thispagestyle{empty}
  \paragraph{Введение.}
    В подсистемах моделирования физики современных автомобильных симуляторов и игровых приложений большое внимание
    уделяется деформации объектов в результате столкновений. Визуально
    правдоподобные деформации повышают реализм игрового процесса, при этом детальная
    физическая достоверность не требуется. Достаточно распространён подход, при
    котором используются предварительно рассчитанные деформированные состояния модели.
    Такой подход имеет следующие недостатки:
    \begin{itemize}
      \item разнообразие возможных деформаций ограничено, что негативно сказывается на реализме;
      \item требуется значительное время на предварительные расчёты после изменения модели, прежде чем
        она может быть загружена в симулятор и протестирована, что создаёт существенные неудобства при разработке.
    \end{itemize}
  \paragraph{Цели.}
    Целью работы является разработка системы моделирования деформаций в реальном
    времени, не требующей значительных предварительных расчётов, которую можно будет легко
    интегрировать в уже имеющийся физический движок.
  \paragraph{Задачи.}
    Для достижения цели необходимо решить следующие задачи:
    \begin{itemize}
      \item разработать и реализовать алгоритм моделирования деформаций неупругих тел, заданных полигональной сеткой;
      \item предоставить интерфейс для двустороннего взаимодействия с приложением, в которое будет
        встроена система, включающего:
        \begin{itemize}
          \item получение извне информации о столкновениях модели и приложенных к ней силах;
          \item сообщение наружу о сильных локальных деформациях;
          \item обновление отображаемой сетки;
        \end{itemize}
      \item предусмотреть в алгоритме возможность использования параллельных вычислений.
    \end{itemize}
\end{document}

