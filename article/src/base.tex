\documentclass[a4paper, 12pt]{article}
  \usepackage{cmap}
  \usepackage[hidelinks,pdftex,unicode]{hyperref}
  \usepackage{mathtext} % для кириллицы в формулах
  \usepackage[T2A]{fontenc}
  \usepackage[utf8]{inputenc}
  \usepackage[english,russian]{babel}
  \usepackage{indentfirst}
  \usepackage{cite}
  \usepackage{amsmath} % для \eqref
  \usepackage{amssymb} % для \leqslant
  \usepackage{amsthm} % для \pushQED
  \usepackage{color} % пока только для TODO:
  \usepackage[pdftex]{graphicx}
  \usepackage{subfig}
  \usepackage{numprint}
  \usepackage[left=3cm,right=2cm,top=2cm,bottom=2cm,bindingoffset=0cm]{geometry}
  \usepackage{datetime}
  \graphicspath{{../img/}{../../img/}}
  \frenchspacing

  \DeclareSymbolFont{T2Aletters}{T2A}{cmr}{m}{it} % кириллица в формулах курсивом

  \addto\captionsrussian{
    \renewcommand\contentsname{Содержание}
    % перекрываю \refname, чтобы список литературы сам добавлял себя в оглавление
    \let\oldrefname\refname
    \renewcommand\refname{\addcontentsline{toc}{section}{\oldrefname}\oldrefname}
  }

  \newcommand{\underscore}[1]{\hbox to#1{\hrulefill}}
  \newcommand{\todo}[1]{\textbf{\textcolor{red}{TODO: #1}}}
  \newcommand{\note}[1]{\textit{Примечание: #1}}
  \newcommand{\eng}[1]{\foreignlanguage{english}{#1}}

  % обёртка с моими настройками поверх figure:
  % \begin{myfigure}{подпись}{fig:label} ... \end{myfigure}
  \newenvironment{myfigure}[2]%
    {\pushQED{\caption{#1} \label{#2}} % push caption & label
     \begin{figure}[!htb]\centering } %
    {  \popQED % pop caption & label
     \end{figure}}

  % вставка картинки: \figure[params]{подпись}{file}
  % создаёт label вида fig:file
  \newcommand{\includefigure}[3][]{
    \begin{myfigure}{#2}{fig:#3}
      \includegraphics[#1]{#3}
    \end{myfigure}
  }

  % вставка subfigure внутри myfigure:
  % \subfigure[params]{подпись}{file}
  \newcommand{\subfigure}[3][]{
    \subfloat[#2]{\label{fig:#3}\includegraphics[#1]{#3}}
  }

  \newcommand{\vect}[1]{\vec{#1}} % единое выделение векторов (стрелкой)
  \newcommand{\matx}[1]{\mathbf{#1}} % единое выделение матриц (полужирным)
  \newcommand{\transposed}{\top} % единый знак транспонирования (U+22A4 down tack)
  \renewcommand{\le}{\leqslant} % <= с наклонной нижней перекладиной
  \renewcommand{\ge}{\geqslant} % >= с наклонной нижней перекладиной

  \linespread{1.3}

  % русские буквы для списков и частей рисунка
  \renewcommand{\theenumii}{(\asbuk{enumii})}
  \renewcommand{\labelenumii}{\asbuk{enumii})}
  \renewcommand{\thesubfigure}{\asbuk{subfigure}}

  \let\oldsection\section
  \renewcommand{\section}{\newpage\oldsection}
)
  \setcounter{tocdepth}{3} % глубина оглавления

  \hyphenation{англ} % убрать перенос в этом сокращении

  % алиас и настройки для numprint
  \newcommand{\num}[1]{\numprint{#1}}
  \npthousandsep{\,}
  \npthousandthpartsep{}
  \npdecimalsign{,}

  \newcommand{\checkdate}[3]{({\Russian дата обращения: \formatdate{#1}{#2}{#3}})}

  \newcommand{\thetitle}{Алгоритм моделирования пластических деформаций в реальном времени для
  приложений виртуальной реальности}
  \newcommand{\theauthora}{И. А. Новиков}
  \newcommand{\theauthorb}{Д. А. Гладкий}

  \author{\theauthora, \theauthorb}
  \title{\thetitle}

  \hypersetup{
    pdfinfo={
      Title = {\thetitle},
      Author = {\theauthora, \theauthorb},
      Subject = {}
    }
  }

\begin{document}

%----------------------- титульник ------------------------

УДК 004.946
\begin{center}
  \textbf{ \large \scshape \thetitle }

  \vspace {0.5cm}

  \textbf{ \theauthora\textsuperscript{1} \theauthorb\textsuperscript{2} }

  \textit{
    \textsuperscript{1}Кубанский государственный университет,\\
    350040, Краснодар, ул. Ставропольская, 149\\
    \textsuperscript{2}Институт автоматики и электрометрии \todo{или нет?}\\
    630090, Новосибирск, пр. Академика Коптюга, 1\\
    Email: nia.afti@gmail.com
  }
\end{center}

%----------------------- аннотация ------------------------

{\small
  Предложен алгоритм моделирования динамики пластических деформаций 3D-объектов, представленных в виде
  полигональных сеток либо произвольного набора точек, который предназначен для применения в целях
  визуализации в приложениях виртуальной реальности. Алгоритм основан на методе
  сопоставления формы (\eng{shape matching}), использующем методы аналитической геометрии для
  выражения трансформации формы объекта. Этот метод расширен дополнительными поправками,
  позволяющими сделать моделирование более стабильным и реалистичным, а также обеспечить эффективную
  визуализацию результатов расчёта путём переноса деформации с модели низкого разрешения,
  используемой при расчётах, на полигональную сетку высокого разрешения, отображаемую на экране
  средствами графического процессора. \todo{цифры}
}

%-------------------------- введение --------------------------
  \paragraph{Введение}
    Приложения виртуальной реальности становятся чрезвычайно рас\-прос\-тра\-нён\-ны\-ми в~наши~дни.  Два
    основных их типа~--- это компьютерные игры и~обучающие симуляторы (тренажёры). Несмотря на
    различные применения, эти два типа приложений имеют много общего, поскольку в~обоих случаях
    требуется создать иллюзию присутствия у~пользователя. Для~этого необходимо как генерировать
    фотореалистичные изображения, так и правдоподобно моделировать физику взаимодействия объектов
    виртуального мира: их движение, столкновения, деформации и~разрушение.

    Моделирование деформаций объектов является актуальной задачей, поскольку кроме приложений
    виртуальной реальности оно находит применение в~системах автоматизированного проектирования, а~также используется
    при~создании спецэффектов к~фильмам. В зависимости от поставленной задачи применяются различные
    подходы. А~именно, в~системах автоматизированного проектирования требуется
    максимальная точность расчётов, в~то~время как в~фильмах и приложениях виртуальной реальности
    требуется лишь визуальная правдоподобность. При~этом, если расчёты для САПР и~спецэффектов могут
    выполняться длительное время, то в~интерактивных приложениях расчёт одного шага вычислений должен
    происходить не~дольше, чем за~промежуток между кадрами, чтобы обеспечивать плавную анимацию
    и отсутствие задержек между воздействием на объект и его деформацией.

    Жёсткое требование быстродействия в приложениях реального времени не~только вынуждает
    использовать упрощённые физические модели деформации тела, менее точные, чем используемые
    в~инженерных расчётах, но и накладывают существенные ограничения на способ представления
    моделируемого объекта. Чаще всего он задаётся дискретным образом: в~виде сетки, решётки или
    несвязанного набора точек, причём время расчётов напрямую (как~правило, линейно
    \cite{mueller-meshless}) зависит от количества точек в таком представлении. Для отображения
    объектов обычно используют высоко детализированные сетки, содержащие вплоть до нескольких
    сотен тысяч и даже миллионов точек. Это значительно превышает максимальное допустимое число точек для~большинства
    алгоритмов \cite{mueller-stable, mueller-meshless, chang-crash} при расчётах в~реальном времени на
    современных настольных компьютерах и компьютерах, используемых в тренажёрах. Из-за этого
    невозможно использовать одно и~то~же представление объекта и для моделирования, и для
    отображения.

  \paragraph{Заключение}\label{sec:conclusion}
    В данной работе ставилась цель разработать систему моделирования деформаций неупругих тел
    в~реальном времени, ориентированную на применение в~приложениях виртуальной реальности. Эта цель
    была достигнута, и были удовлетворены поставленные в разделе~\ref{sec:task} требования, включая
    высокое быстродействие и удобный для интеграции интерфейс, а~также применение параллельных
    вычислений и отказ от использования предварительно рассчитанных деформаций. Более того,
    использование графического процессора позволило значительно улучшить быстродействие.

    Предусмотрено тестирование системы с~помощью модульных тестов. Также было разработано простое интерактивное
    приложение, использующее возможности системы, чтобы моделировать деформаций предварительно
    заданного объекта. С помощью этого приложения была протестирована правдоподобность деформаций и
    исследованы быстродействие и зависимость характера деформаций от параметров алгоритма.
    При правильном подборе этих параметров система моделирует достаточно реалистичные изображения
    для применения её в приложениях виртуальной реальности.
    Также были исследованы возможности дальнейшего развития и улучшения системы. В частности, планируется разработка
    дополнительных инструментов, упрощающих использование системы разработчиками приложений
    виртуальной реальности.

  \begin{flushleft}
    \bibliographystyle{../../biblio/ugost2003} % Или использовать ugost2008 для нового ГОСТа
    \bibliography{../../biblio/my}
  \end{flushleft}
\end{document}

