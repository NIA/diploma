\documentclass[a4paper,11pt]{report}
  \usepackage{cmap}
  \usepackage[english,russian]{babel}
  \usepackage[utf8]{inputenc}
  \usepackage{indentfirst}
  \addto\captionsrussian{ \renewcommand\chaptername{Раздел} }
  \frenchspacing

  \author{И.\,А.\,Новиков, кафедра АФТИ ФФ НГУ, гр.\,7305\\ Руководитель: Д.\,А.\,Гладкий, Софтлаб-Нск}
  \title{<<Система моделирования деформаций неупругих тел в реальном времени>>\\\itshape Октябрьский отчёт}
\begin{document}
  \maketitle
  \tableofcontents
  \chapter*{Введение}
  \addcontentsline{toc}{chapter}{Введение}
  \chapter{Описание предметной области}
  \chapter{Постановка задачи}
  \chapter{Обзор литературы}
    \cite[Physically Based Deformable Models...]{muller-physmodels}. Статья представляет собой
      обзор распространённых фиических методов моделирования деформаций. Приводится классификация
      методов и вкратце описывается основная идея каждого метода, а также его особенности,
      преимущества и недостатки.

    \cite[Meshless Deformations...]{muller-meshless}. В статье описывается алгоритм
      моделирования деформаций тела, заданного набором вершин, не требующий информации о связях
      между ними. Алгоритм геометрический, то есть прямо не моделирует физические законы, но путём
      наложения дополнительных ограничений можно добиться правдоподобно выглядящих результатов.
      Преимуществом алгоритма является то, что отдельные вершины обрабатываются независимо,
      благодаря чему расчёты для них можно производить параллельно.

    \cite[Stable Real-Time Deformations]{muller-stable}. В статье описывается иной алгоритм
      моделирования деформаций, основанный на методе конечных элементов. Основной идеей является то,
      что непрерывные характеристики деформируемого тела вычисляются путём интерполяции их значений
      на элементах конечного размера. Этот алгоритм обеспечивает более реалистичные деформации, чем
      описанный в \cite{muller-meshless}, но требует значительно больше времени для расчётов. Кроме
      того, объект нужно задавать объёмной тетраэдрическай сеткой, в то время как в приложениях
      виртуальной реальности и играх чаще используются поверхностные сетки.

    \cite[Position Based Dynamics]{muller-position-dynamics}. Авторы этой статьи предлагают
      альтернативный подход к моделированию деформаций мягких тел (в частности, ткани): вместо
      вычисления действия на тело набора сил, в алгоритме используются только позиции и скорости
      точек, составляющих тело, на которые накладываются ограничения (англ. {\English constraints}),
      обеспечивающие требуемые свойства материала, а также реакцию на столкновения. В статье
      предлагаются некоторые идеи, которые могут быть полезны и при традиционном подходе, например,
      способ расчёта демпфирования.

    \cite[Deforming a High-Resolution Mesh...]{visser-mapping}. Статья посвящена проблеме
      использования сеток разной степени детализации для моделирования физики и для отображения на
      экране, возникающей из-за того, что максимальное число вершин в сетке, при котором возможен
      расчёт физики в реальном времени, меньше, чем необходимое для отображения деталей объекта.
      Необходимо, чтобы сетка высокого разрешения повторяла деформации сетки низкого разрешения, но
      чтобы при этом форма направляющей сетки не прослеживалась по движению отображаемой.

    \cite[Efficient Mesh Deformation Using Tetrahedron Control Mesh]{huang-control-mesh}.
      В статье предлагается решение задачи деформации сложного объекта, заданной деформацией простой
      контрольной сетки. Идеи представленного алгоритма интерполяции могут быть применены для
      обеспечения соответствия между низко детализированной физической моделью и высоко детализированной
      сеткой для отображения, как в \cite{visser-mapping}, но так как в этой статье ставится другая
      задача, предложенный алгоритм более трудоемкий, и допускает лишь несколько сотен вершин в
      контрольной сетке.

    \cite[Numerical Recipes...]{fortran-jacobi}. В главе {\English ``Jacobi Transformations of
      a Symmetric Matrix''} этой книги описываются <<преобразования Якоби>>, каждое из которых
      представляет собой отдельный шаг приближенного итеративного алгоритма диагонализации
      симметричной матрицы. Этот алгоритм может быть применен для диагонализации матриц, возникающих
      в алгоритме моделирования деформаций, описанном в \cite{muller-meshless}.
      
    \cite[Frobenius Iteration for the Matrix Polar Decomposition]{hp-polar}. В статье
      предлагается итеративный алгоритм для приближенного вычисления полярного разложения
      произвольной матрицы. Задача нахождения полярного разложения возникает в \cite{muller-meshless},
      так что описанный алгоритм может быть применён вместо предложенного в \cite{muller-meshless},
      будучи более эффективным и надёжным.

    \cite[What to Look for When Evaluating Middleware for Integration]{gems-middleware}. В
      этой статье из сборника {\English Game Engine Gems} автор предлагает критерии, которыми
      следует руководствоваться при выборе промежуточного ПО (англ. {\English middleware}) для интеграции в
      коммерческое приложение (в частности --- игровой движок). Следование этим критериям
      значительно упрощает интеграцию, поэтому нужно стремиться к тому, чтобы разрабатываемая
      система удовлетворяла им.

  \begin{thebibliography}{9}
  \addcontentsline{toc}{chapter}{Литература}
    \bibitem{muller-physmodels}
      A. Nealen, M. Müller, R. Keiser, E. Boxerman, M. Carlson.
      \newblock Physically Based Deformable Models in Computer Graphics.
      \newblock --- Computer Graphics Forum, Vol. 25, issue 4, Wiley-Blackwell, 2006.
      \newblock --- с.~809--836.

    \bibitem{muller-meshless}
      M. Müller, B. Heidelberger, M. Teschner, M. Gross.
      \newblock Meshless Deformations Based on Shape Matching.
      \newblock --- Proceedings of SIGGRAPH'05, Los Angeles, USA, 2005.
      \newblock --- с.~471--478.

    \bibitem{muller-stable}
      M. Müller, J. Dorsey, L. McMillan, R. Jagnow, B. Cutler.
      \newblock Stable Real-Time Deformations.
      \newblock --- Proceedings of ACM SIGGRAPH Symposium on Computer Animation (SCA), San Antonio,
                    USA, 2002.
      \newblock --- с.~49--54

    \bibitem{muller-position-dynamics}
      M. Müller, B. Heidelberger, M. Hennix, J. Ratcliff.
      \newblock Position Based Dynamics.
      \newblock --- Proceedings of Virtual Reality Interactions and Physical Simulations (VRIPhys),
                    Madrid, Spain, 2006.
      \newblock --- с.~71--80.

    \bibitem{visser-mapping}
      Hans de Visser, Olivier Comas, David Conlan, Sébastien Ourselin, Josh Passenger, Olivier Salvado.
      \newblock Deforming a High-Resolution Mesh in Real-Time by Mapping onto a Low-Resolution Physical Model.
      \newblock --- Lecture Notes in Computer Science, Vol. 5104/2008, Springer Science+Business Media,
                    USA, 2008.
      \newblock --- с.~135--146.

    \bibitem{huang-control-mesh}
      Jin Huang, Lu Chen, Xinguo Liu, Hujun Bao.
      \newblock Efficient Mesh Deformation Using Tetrahedron Control Mesh.
      \newblock --- Proceedings of the 2008 ACM symposium on Solid and Physical Modeling, New York,
                    USA, 2008.
      \newblock --- с.~241--247.

    \bibitem{fortran-jacobi}
      William Press, Brian Flannery, Saul Teukolsky.
      \newblock Numerical Recipes in FORTRAN 77, Vol. 1.
      \newblock --- Cambridge University Press, 1992.
      \newblock --- с.~456--462.

    \bibitem{hp-polar}
      Augustin A. Dubrulle.
      \newblock Frobenius Iteration for the Matrix Polar Decomposition
      \newblock --- HP Labs Technical Reports, 1994

    \bibitem{gems-middleware}
      Jason Hughes.
      \newblock What to Look for When Evaluating Middleware for Integration.
      \newblock --- Game Engine Gems Vol. 1, Jones and Barlett Publishers, USA, 2010.
      \newblock --- с.~3-10.
  \end{thebibliography}
\end{document}

